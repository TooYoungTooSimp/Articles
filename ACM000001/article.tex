% Options for packages loaded elsewhere
\PassOptionsToPackage{unicode=true}{hyperref}
\PassOptionsToPackage{hyphens}{url}
%
\documentclass[
]{article}
\usepackage{lmodern}
\usepackage{amssymb,amsmath}
\usepackage{ifxetex,ifluatex}
\ifnum 0\ifxetex 1\fi\ifluatex 1\fi=0 % if pdftex
  \usepackage[T1]{fontenc}
  \usepackage[utf8]{inputenc}
  \usepackage{textcomp} % provides euro and other symbols
\else % if luatex or xelatex
  \usepackage{unicode-math}
  \defaultfontfeatures{Scale=MatchLowercase}
  \defaultfontfeatures[\rmfamily]{Ligatures=TeX,Scale=1}
\fi
% Use upquote if available, for straight quotes in verbatim environments
\IfFileExists{upquote.sty}{\usepackage{upquote}}{}
\IfFileExists{microtype.sty}{% use microtype if available
  \usepackage[]{microtype}
  \UseMicrotypeSet[protrusion]{basicmath} % disable protrusion for tt fonts
}{}
\makeatletter
\@ifundefined{KOMAClassName}{% if non-KOMA class
  \IfFileExists{parskip.sty}{%
    \usepackage{parskip}
  }{% else
    \setlength{\parindent}{0pt}
    \setlength{\parskip}{6pt plus 2pt minus 1pt}}
}{% if KOMA class
  \KOMAoptions{parskip=half}}
\makeatother
\usepackage{xcolor}
\IfFileExists{xurl.sty}{\usepackage{xurl}}{} % add URL line breaks if available
\IfFileExists{bookmark.sty}{\usepackage{bookmark}}{\usepackage{hyperref}}
\hypersetup{
  pdftitle={A brief introduction to pb\_ds for ICPC},
  pdfauthor={Yuchen Lei},
  hidelinks,
}
\urlstyle{same} % disable monospaced font for URLs
\usepackage[a4paper,scale=0.75]{geometry}
\usepackage{color}
\usepackage{fancyvrb}
\newcommand{\VerbBar}{|}
\newcommand{\VERB}{\Verb[commandchars=\\\{\}]}
\DefineVerbatimEnvironment{Highlighting}{Verbatim}{commandchars=\\\{\}}
% Add ',fontsize=\small' for more characters per line
\newenvironment{Shaded}{}{}
\newcommand{\AlertTok}[1]{\textcolor[rgb]{1.00,0.00,0.00}{\textbf{#1}}}
\newcommand{\AnnotationTok}[1]{\textcolor[rgb]{0.38,0.63,0.69}{\textbf{\textit{#1}}}}
\newcommand{\AttributeTok}[1]{\textcolor[rgb]{0.49,0.56,0.16}{#1}}
\newcommand{\BaseNTok}[1]{\textcolor[rgb]{0.25,0.63,0.44}{#1}}
\newcommand{\BuiltInTok}[1]{#1}
\newcommand{\CharTok}[1]{\textcolor[rgb]{0.25,0.44,0.63}{#1}}
\newcommand{\CommentTok}[1]{\textcolor[rgb]{0.38,0.63,0.69}{\textit{#1}}}
\newcommand{\CommentVarTok}[1]{\textcolor[rgb]{0.38,0.63,0.69}{\textbf{\textit{#1}}}}
\newcommand{\ConstantTok}[1]{\textcolor[rgb]{0.53,0.00,0.00}{#1}}
\newcommand{\ControlFlowTok}[1]{\textcolor[rgb]{0.00,0.44,0.13}{\textbf{#1}}}
\newcommand{\DataTypeTok}[1]{\textcolor[rgb]{0.56,0.13,0.00}{#1}}
\newcommand{\DecValTok}[1]{\textcolor[rgb]{0.25,0.63,0.44}{#1}}
\newcommand{\DocumentationTok}[1]{\textcolor[rgb]{0.73,0.13,0.13}{\textit{#1}}}
\newcommand{\ErrorTok}[1]{\textcolor[rgb]{1.00,0.00,0.00}{\textbf{#1}}}
\newcommand{\ExtensionTok}[1]{#1}
\newcommand{\FloatTok}[1]{\textcolor[rgb]{0.25,0.63,0.44}{#1}}
\newcommand{\FunctionTok}[1]{\textcolor[rgb]{0.02,0.16,0.49}{#1}}
\newcommand{\ImportTok}[1]{#1}
\newcommand{\InformationTok}[1]{\textcolor[rgb]{0.38,0.63,0.69}{\textbf{\textit{#1}}}}
\newcommand{\KeywordTok}[1]{\textcolor[rgb]{0.00,0.44,0.13}{\textbf{#1}}}
\newcommand{\NormalTok}[1]{#1}
\newcommand{\OperatorTok}[1]{\textcolor[rgb]{0.40,0.40,0.40}{#1}}
\newcommand{\OtherTok}[1]{\textcolor[rgb]{0.00,0.44,0.13}{#1}}
\newcommand{\PreprocessorTok}[1]{\textcolor[rgb]{0.74,0.48,0.00}{#1}}
\newcommand{\RegionMarkerTok}[1]{#1}
\newcommand{\SpecialCharTok}[1]{\textcolor[rgb]{0.25,0.44,0.63}{#1}}
\newcommand{\SpecialStringTok}[1]{\textcolor[rgb]{0.73,0.40,0.53}{#1}}
\newcommand{\StringTok}[1]{\textcolor[rgb]{0.25,0.44,0.63}{#1}}
\newcommand{\VariableTok}[1]{\textcolor[rgb]{0.10,0.09,0.49}{#1}}
\newcommand{\VerbatimStringTok}[1]{\textcolor[rgb]{0.25,0.44,0.63}{#1}}
\newcommand{\WarningTok}[1]{\textcolor[rgb]{0.38,0.63,0.69}{\textbf{\textit{#1}}}}
\usepackage{longtable,booktabs}
% Allow footnotes in longtable head/foot
\IfFileExists{footnotehyper.sty}{\usepackage{footnotehyper}}{\usepackage{footnote}}
\makesavenoteenv{longtable}
\setlength{\emergencystretch}{3em} % prevent overfull lines
\providecommand{\tightlist}{%
  \setlength{\itemsep}{0pt}\setlength{\parskip}{0pt}}
\setcounter{secnumdepth}{-\maxdimen} % remove section numbering
% Redefines (sub)paragraphs to behave more like sections
\ifx\paragraph\undefined\else
  \let\oldparagraph\paragraph
  \renewcommand{\paragraph}[1]{\oldparagraph{#1}\mbox{}}
\fi
\ifx\subparagraph\undefined\else
  \let\oldsubparagraph\subparagraph
  \renewcommand{\subparagraph}[1]{\oldsubparagraph{#1}\mbox{}}
\fi

% Set default figure placement to htbp
\makeatletter
\def\fps@figure{htbp}
\makeatother

\pagenumbering{gobble}

\title{A brief introduction to pb\_ds for ICPC}
\author{Yuchen Lei}
\date{\today}

\begin{document}
\maketitle
\begin{abstract}
pb\_ds, also known as Policy-Based Data Structures, is a g++ specific
library of policy-based elementary data structures. In the rest of this
article, we will take a glance at this library.
\end{abstract}

{
\setcounter{tocdepth}{3}
\tableofcontents
}
\clearpage
\pagenumbering{arabic}

\hypertarget{heap}{%
\subsection{Heap}\label{heap}}

In particular C++, when we need a heap, we will include queue and just
use priority\_queue. In most situations, naive priority\_queue does fit
in use: \(O(\log{n})\) push and pop, and \(O(1)\) top operation. But in
the most tough situation (for example data maker is yswang), we may need
faster heap to accomplish such mission.

Lots of different kinds of heap are provided in pb\_ds, in order to use
them, just include \texttt{ext/pb\_ds/priority\_queue.hpp} and use
\texttt{\_\_gnu\_pbds::priority\_queue}:

\begin{Shaded}
\begin{Highlighting}[]
\KeywordTok{template}\NormalTok{<}
    \KeywordTok{typename}\NormalTok{ Value_Type,}
    \KeywordTok{typename}\NormalTok{ Cmp_Fn = }\BuiltInTok{std::}\NormalTok{less<Value_Type>,}
    \KeywordTok{typename}\NormalTok{ Tag = pairing_heap_tag,}
    \KeywordTok{typename}\NormalTok{ Allocator = }\BuiltInTok{std::}\NormalTok{allocator<}\DataTypeTok{char}\NormalTok{> >}
\KeywordTok{class}\NormalTok{ priority_queue;}
\end{Highlighting}
\end{Shaded}

The biggest different is the \texttt{Tag} template argument, pb\_ds use
this argument to determine which kind of head to use actually. As for
now, we can select \texttt{Tag} from \texttt{pairing\_heap\_tag},
\texttt{binary\_heap\_tag}, \texttt{binomial\_heap\_tag},
\texttt{rc\_binomial\_heap\_tag}, or \texttt{thin\_heap\_tag}. Here is a
simple table which compares these heaps.

\begin{longtable}[]{@{}lllll@{}}
\toprule
& Push & Pop & Modify/Erase & Join\tabularnewline
\midrule
\endhead
pairing\_heap\_tag & \(O(1)\) & \(O(\log{n})\)/\(O(n)\) &
\(O(\log{n})\)/\(O(n)\) & \(O(1)\)\tabularnewline
binary\_heap\_tag & \(O(\log{n})\)/\(O(n)\) & \(O(\log{n})\)/\(O(n)\) &
\(O(1)\) & \(O(1)\)\tabularnewline
binomial\_heap\_tag & \(O(1)\)/\(O(\log{n})\) & \(O(\log{n})\) &
\(O(\log{n})\) & \(O(\log{n})\)\tabularnewline
rc\_binomial\_heap\_tag & \(O(1)\) & \(O(\log{n})\) & \(O(\log{n})\) &
\(O(\log{n})\)\tabularnewline
thin\_heap\_tag & \(O(1)\) & \(O(\log{n})\)/\(O(n)\) & \(O(\log{n})\) &
\(O(\log{n})\)/\(O(n)\)\tabularnewline
\bottomrule
\end{longtable}

Unlike STL's priority\_queue, priority\_queue in pb\_ds supports more
opeartions such as remove, modify and join.

\end{document}
